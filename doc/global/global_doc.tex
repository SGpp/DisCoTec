 \documentclass[11pt]{article}
\usepackage{graphicx}
\usepackage{wrapfig}

\usepackage{amssymb}

\textwidth = 6.5 in
\textheight = 9 in
\oddsidemargin = 0.0 in
\evensidemargin = 0.0 in
\topmargin = 0.0 in
\headheight = 0.0 in
\headsep = 0.0 in
\parskip = 0.0in
\parindent = 0.4in

\newtheorem{theorem}{Theorem}
\newtheorem{corollary}[theorem]{Corollary}
\newtheorem{definition}{Definition}



\begin{document}

\baselineskip=24pt

\begin{titlepage}

\centerline{\bf Global GENE tutorial }


\bigskip

\medskip

\baselineskip=24pt


\noindent{\bf I. The `nonlocal' namelist: parameters specific to global GENE}

\begin{description}
\item[\texttt{l\_buffer\_size [real]:}] (nonlocal namelist) portion of the simulation domain devoted to the lower buffer zone.  A typical value would be $\lesssim 0.1$.
\item[\texttt{u\_buffer\_size [real]:}] (nonlocal namelist) portion of the simulation domain devoted to the upper buffer zone.  A typical value would be $\lesssim 0.1$.
\item[\texttt{lcoef\_krook [real]:}] (nonlocal namelist) amplitude for the lower krook operator.  A typical value would be $\sim 1.0$.
\item[\texttt{ucoef\_krook [real]:}] (nonlocal namelist) amplitude for the upper krook operator.  A typical value would be $\sim 1.0$.
\item[\texttt{lpow\_krook [int]:}] (nonlocal namelist) power for the lower krook operator.  Default: 4.
\item[\texttt{upow\_krook [int]:}] (nonlocal namelist) power for the upper krook operator.  Default: 4. 
\item[\texttt{ck\_heat [real]:}] (nonlocal namelist) [Needs description].  A typical value should be somewhat smaller than the linear growth rates.
\item[\texttt{ck\_part [real]:}] (nonlocal namelist) [Needs description].  A typical value should be somewhat smaller than the linear growth rates.
\item[\texttt{ck\_heat\_smooth [real 0.0]:}] (nonlocal namelist) width of a radial smoothing window for the krook-type heat source relativ to the total radial simulation domain. Default: no smoothing
\item[\texttt{ck\_part\_smooth [real 0.0]:}] (nonlocal namelist) width of a radial smoothing window for the krook-type particle source relativ to the total radial simulation domain. Default: no smoothing

\end{description}

\noindent{\bf I. Other parameters to be aware of when using global GENE}

\begin{description}
\item[\texttt{x\_local [logical]:}] (general namelist) flag for running radially global version of GENE.
\item[\texttt{y\_local [logical]:}] (general namelist) flag for running y-global version of GENE.
\item[\texttt{x0 [real]:}] (box namelist) radial position of the center of the simulation domain.
\item[\texttt{kymin [real]:}] (box namelist) minimum binormal wavenumber.  If \texttt{kymin} is specified then \texttt{kymin} will be automatically adjusted corresponding to the \texttt{n0\_global} (which must be an integer) which produces the \texttt{kymin} value closest to the input value.  Alternatively, if \texttt{n0\_global} is specified then \texttt{kymin} is automatically calculated.
\item[\texttt{n0\_global [int]:}] (box namelist) toroidal mode number with \texttt{n0\_global}=1 representing a full-torus simulation domain.  If \texttt{n0\_global} is specified, \texttt{kymin} is calculated automatically, and vice versa if \texttt{kymin} is specified.  
\item[\texttt{lx [real]:}] (box namelist) radial box size in units of the reference gyroradius.  The user must ensure that this is smaller than the radial extent of the input profiles and magnetic geometry file.
\item[\texttt{lv [real]:}] (box namelist) parallel velocity box size.  The velocity coordinates are normalized to the thermal velocity at the center of the box.  Thus \texttt{lv} must be increased corresponding to the highest temperature region of the box.  Note that \texttt{lv} varies as the square root of the temperature.  
\item[\texttt{lw [real]:}] (box namelist) $\mu$ box size.  As with \texttt{lv}, the $\mu$ box size must be increased to account for the temperature profile variation.  Note that \texttt{lw} varies linearly with the temperature.
\item[\texttt{nv0 [int]:}] (box namelist) number of parallel velocity grid-points.  This must be increased such that the velocity dynamics in the low-temperature region of the radial domain (where the distribution function is concentrated at low velocities) are still well resolved.
\item[\texttt{nw0 [int]:}] (box namelist) number of $\mu$ grid-points.  This must be increased such that the velocity dynamics in the low-temperature region of the radial domain (where the distribution function is concentrated at low velocities) are still well resolved.
\item[\texttt{istep\_vsp [int]:}] (in\_out namelist) number of time iterations between activation of vsp diagnostic.  Care must be taken with high velocity space resolution that vsp.dat files do not consume exorbitant amounts of memory.
\item[\texttt{istep\_prof [int]:}] (in\_out namelist) number of time iterations between activation of the profile diagnostic.  [Need explanation of profile diagnostic].
\item[\texttt{hyp\_x [int]:}] (general namelist) amplitude of radial hyper-diffusion.  Some hyperdiffusion is necessary to eliminate small scale radial fluctuations.  A typical value would be $\sim0.5$.  
\item[\texttt{rhostar [real]:}] (geometry namelist) $\rho_{ref}/a$, where $a$ is the minor radius. 
\item[\texttt{minor\_r [real]:}] (geometry namelist) normalized minor radius, $a/L_{ref}$. 
\item[\texttt{major\_r [real]:}] (geometry namelist) normalized major radius, $R/L_{ref}$. 
\item[\texttt{mag\_prof [logical]:}] (geometry namelist) use radially dependent magnetic profiles . The \texttt{q\_coeffs} input parameter is then used to prescribe profiles in terms of a fifth order polynomial (unless numerical input from EFIT, CHEASE or GIST files is used). Currently, this switch must be set to true in order to run the x-global code. 
\item[\texttt{q\_coeffs [real(6)]:}] (geometry namelist) each element of the array is a coefficient for a fifth order polynomial used to prescribe the magnetic profile, i.e., the first element is the coefficient of the $x^0$ term, the second element is the coefficient of the linear term, etc. 
\item[\texttt{prof\_type [int]:}] (species namelist) tells GENE how to prescribe temperature and density profiles.  Acceptable values are $[-3,5]$.  

$-3:-1\Rightarrow$ file input (-1 for standard GENE profiles).  

$0\Rightarrow$ local. 

$1:5\Rightarrow$ analytical profiles. 

\end{description}

\noindent{\bf III. Comments on parallelization}

As with the local version, species parallelization is most efficient and $\mu$ parallelization is also effective when collisions are not being used.  Even with collisions, $\mu$ parallelization is often competitive with other coordinates.  For x-global, radial parallelization is quite efficient.  Parallelization in the y-coordinate is not efficient and cannot be used in conjunction with radial parallelization.  Parallel velocity parallelization has the drawback that it does not reduce the size of the gyroaveraging matrix (which is independent of $v_{||}$) on each processor. 





\end{titlepage}

\setcounter{page}{2}

\begin{thebibliography}{99}

\bibitem{reference} Add references.


\end{thebibliography}

%\newpage


%\clearpage
%\newpage
%\begin{figure}
%\includegraphics[width=15.0cm]{ky_combo.ps}
%\caption{}
%\label{figure:ky_combo}
%\end{figure}






\end{document}
