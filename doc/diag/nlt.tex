 \documentclass[11pt]{article}
\usepackage{graphicx}
\usepackage{wrapfig}

\usepackage{amssymb}

\textwidth = 6.5 in
\textheight = 9 in
\oddsidemargin = 0.0 in
\evensidemargin = 0.0 in
\topmargin = 0.0 in
\headheight = 0.0 in
\headsep = 0.0 in
\parskip = 0.0in
\parindent = 0.4in

\newtheorem{theorem}{Theorem}
\newtheorem{corollary}[theorem]{Corollary}
\newtheorem{definition}{Definition}



\begin{document}
%\{LLNL-JRNL-463759}

\baselineskip=24pt

\begin{titlepage}

\centerline{\bf Gyrokinetic nonlinear energy transfer functions }


\bigskip

%\centerline{D. R. Hatch - Aug. 25, 2011}
%\centerline{\it davidhat@ipp.rzg.mpg.de}
%\centerline{\it UW and IPP, finish} 
%\centerline{D. del-Castillo-Negrete}
%\centerline{\it delcastillod@ornl.gov}
%\centerline{\it Oak Ridge National Laboratory } 
%\centerline{\it Oak Ridge TN, 37831-8071} 
%\centerline{P. W. Terry}
%\centerline{\it UW-Madison - finish} 
%\centerline{D. R. Hatch$^{1,2}$, D. del-Castillo-Negrete$^3$, P. W. Terry$^1$ }
%\centerline{\it $^1$University of Wisconsin-Madison, Madison, Wisconsin 53706}
%\centerline{\it $^2$Max-Planck-Institut f\"{u}r Plasmaphysik, EURATOM Association, 85748 Garching, Germany}
%\centerline{\it $^3$Oak Ridge National Laboratory, Oak Ridge TN, 37831-8071} 

\medskip

%\begin{abstract}

\baselineskip=24pt

\noindent{\bf I. Derivation of Nonlinear Transfer Function}


Here the gyrokinetic nonlinear transfer function is derived using notation from~\cite{F_thesis}.  The gyrokinetic free energy is 
\begin{equation}
E_{k}=\frac{1}{2}\sum_j \pi B_0 n_{0j} T_{0j} \int dz dv_{||} d\mu J(z) \frac{f_{k_j}^2}{F_{0j}} + \frac{1}{2}D(k_{\perp})\phi_k^2+\frac{k_{\perp}}{\beta}A_{||_k}^2.
\label{free_energy}
\end{equation}
This is derived by use of the following `energy operator':
\begin{equation}
\mathcal{E}_{k}[F,G]=\sum_j \pi B_0 n_{0j} T_{0j} \int dz dv_{||} d\mu J(z) 1/F_{0j}\Gamma_j[F]^*G + c.c.,
\label{nl_operator}
\end{equation}
where $\Gamma_j[g]=g_j+\frac{q_j F_{0j}}{T_{0j}}(\bar{\phi}_j-v_{Tj}v_{||}\bar{A}_{||_j})$, and $\phi$ and $A_{||}$ are moments of $g$.
The energy evolution equation is derived by applying the energy operator to the gyrokinetic equation:
\begin{equation}
\frac{\partial E_k}{\partial t}=\mathcal{E}_{k}\left[g_j,\frac{\partial g_j}{\partial t}=\mathcal{L}[g_j]+\mathcal{N}[g_j]\right].
\label{nl_operator}
\end{equation}
When this expression is summed over $k$, the nonlinearity and most of the linear terms vanish.  The only remaining terms are the net sources and sinks wich can be easily related to the heat flux and collisional (and numerical) dissipation.  However, we are currently interested in the nonlinear transfer term which does not vanish at each wavenumber, but rather transfers energy conservatively to other scales, thereby saturating the instability.  The nonlinear operator is as follows:
\begin{equation}
\mathcal{N}[g_j]=\displaystyle \sum_{\vec{k'_{\perp}}}(k'_xk_y-k_xk'_y)    \chi_j(\vec{k'_{\perp}})    g_j(\vec{k_{\perp}}-\vec{k'_{\perp}}).
\label{nl_operator}
\end{equation}
When multiplied by $C_{0j} \Gamma_j^*$ we have (dropping the $k$ summation, and using $C_{0j}\equiv \pi B_0 n_{0j} T_{0j}/F_{0j)$,
\begin{equation}
\nonumber T^{k,k'}\equiv C_{0j} \Gamma_{j,k}^* \mathcal{N}^{k,k'}[g_j]=\\
\label{nl_energy1}
\end{equation}
\begin{equation}
 C_{0j} (k'_xk_y-k_xk'_y)   g_j(-\vec{k_{\perp}}) \chi_j(\vec{k'_{\perp}})    g_j(\vec{k_{\perp}}-\vec{k'_{\perp}})\\
\label{nl_energy2}
\end{equation}
\begin{equation}
+C_{0j}\frac{q_jF_{0j}}{T_{0j}} (k'_xk_y-k_xk'_y)   \chi_j(-\vec{k_{\perp}}) \chi_j(\vec{k'_{\perp}})    g_j(\vec{k_{\perp}}-\vec{k'_{\perp}}),
\label{nl_energy3}
\end{equation}
\begin{equation}
\nonumber =T_g^{k,k'}+T_\chi^{k,k'}
\label{nl_energy4}
\end{equation}
where the reality constraint, $\hat{f}_k^*=\hat{f}_{-k}$, has been used.  

The utility of the nonlinear transfer function is that, for a given wavenumber, one can see where energy is nonlinearly transfered \emph{to} or \emph{from} in $k$-space.  To see this for $T_\chi$, note the property $T_\chi^{k,k'}=-T_\chi^{k',k}$.  Here the coupling coefficient reverses sign,   
\begin{equation}
(k'_xk_y-k_xk'_y)\rightarrow-(k'_xk_y-k_xk'_y),
\label{coupling1}
\end{equation}
while the remaining terms are unchanged:
\begin{eqnarray}
\nonumber &\chi^*_j(\vec{k_{\perp}}) \chi_j(\vec{k'_{\perp}})    g_j(\vec{k_{\perp}}-\vec{k'_{\perp}}) \rightarrow \chi^*_j(\vec{k'_{\perp}}) \chi_j(\vec{k_{\perp}}) g_j(\vec{k'_{\perp}}-\vec{k_{\perp}})&\\
&=\chi_j(\vec{k_{\perp}}) \chi^*_j(\vec{k'_{\perp}})    g^*_j(\vec{k_{\perp}}-\vec{k'_{\perp}})=\chi^*_j(\vec{k_{\perp}}) \chi_j(\vec{k'_{\perp}})    g_j(\vec{k_{\perp}}-\vec{k'_{\perp}}),
\label{}
\end{eqnarray}
where the final result uses the fact that the energy equation is real and thus equal to its complex conjugate (note the complex conjugate in Eq. 2 ).

%To account for Eq.~\ref{nl_energy1}, apply the following coordinate transformation (which leaves the sum unchanged):
Now consider $T_g$, after changing the summation variable to $k''\equiv k-k'  $ so that $T_g$ becomes,
\begin{equation}
T_g^{k,k''}=(k_xk''_y-k''_xk_y)   g_j(-\vec{k_{\perp}}) \chi_j(\vec{k_{\perp}}-\vec{k''_{\perp}})    g_j(\vec{k''_{\perp}}).
\label{}
\end{equation}
Now the same property, $T_g^{k,k''}=-T_g^{k'',k}$, is also observed.  For convenience, the notation in the previous expression is reverted to $T_g^{k,k'}$ so that the final expression for the nonlinear transfer function is,
\begin{equation}
T^{k,k'}=C_{0j}(k'_xk_y-k_xk'_y) \left[  -g_j(-\vec{k_{\perp}}) \chi_j(\vec{k_{\perp}}-\vec{k'_{\perp}})    g_j(\vec{k'_{\perp}})+\frac{q_jF_{0j}}{T_{0j}}   \chi_j(-\vec{k_{\perp}}) \chi_j(\vec{k'_{\perp}})    g_j(\vec{k_{\perp}}-\vec{k'_{\perp}}) \right ]
\label{}
\end{equation}


Because of the property,  
\begin{equation}
T^{k,k'}=-T^{k',k},
\label{}
\end{equation}
the nonlinear transfer function shows where energy for the wavenumber $k$ is nonlinearly transferred to or from.  


\centerline{\bf Nonlinear transfer functions in the GENE diagnostics }
In GENE, the quantity
\begin{equation}
\int dz dv_{||} d\mu J(z) T^{k,k'}
\label{}
\end{equation}
can be calculated and output for selected values of $k$ using the following input parameters: 
\begin{description}
\item[\texttt{istep\_nlt [int]:}] specifies the number of time steps between calculation of nonlinear transfer functions.
\item[\texttt{num\_nlt\_modes [int]:}] number of $k_x,k_y$ modes for which to calculate nonlinear transfer functions (maximum of twenty).
\item[\texttt{kx\_nlt\_ind [int(20)]:}] twenty element integer array specifying the $k_x$ indices for which to calculate nonlinear transfer functions.
\item[\texttt{ky\_nlt\_ind [int(20)]:}] twenty element integer array specifying the $k_y$ indices for which to calculate nonlinear transfer functions.  These indices match with the kx\_nlt\_ind values to define the $k_x,k_y$ pairs.
\end{description}
Data is output to `nlt.dat' and `nlt\_info.dat'.  The file `nlt.dat' contains a time series of nonlinear transfer functions as functions of $k'_x,k'_y$.  The file `nlt\_info.dat' contains time traces (for all selected $k_x,k_y$ pairs) of 1) the total (sum over $k'_x,k'_y$) nonlinear transfer function, 2) the total linear part of the energy equation, 3) the total energy, and 4) a time derivative of 3) which can be compared with the sum of 1) and 2).  These four quantities are output for all the desired sets of $k_x,k_y$ modes.   

This can then be analyzed and visualized using the `nlt' tab in the GENE diagnostics tool.





%This implies that the expression in Eq.~\ref{nl_energy2} is equal to the negative of itself - in other words it is zero.
%Likewise, in order to address the expression in Eq.~\ref{nl_energy3}, consider the following coordinate transformation (which leaves the sum unchanged):
%\begin{equation}
%\nonumber \vec{k}\rightarrow -\vec{k'}\\
%\end{equation}
%\begin{equation}
%\nonumber \vec{k'}\rightarrow -\vec{k}\\
%\end{equation}
%\begin{equation}
%\vec{k}-\vec{k'}\rightarrow \vec{k}-\vec{k'}.
%\label{coord_transform2}
%\end{equation}

























%\begin{equation}
%W[F,G]\equiv \displaystyle \sum_{k_x,k_y} \sum_j \int dzdv_{||}d\mu J(z) \frac{\pi B_0 n_0 T_{0j}}{F_{0j}}(F+\frac{q_j F_{0j}}{T_{0j}} \chi_j[F])^*G,
%W[F,G]\equiv\left \langle \frac{\pi B_0 n_0 T_{0j}}{F_{0j}}\Gamma_j[F]^*G \right \rangle,
%\label{app_energy_operator}
%\end{equation}

%\end{abstract}

\end{titlepage}

\setcounter{page}{2}

\begin{thebibliography}{99}

\bibitem{F_thesis} F. Merz, Ph.D. Thesis, Universit\"at M\"unster, (2008).


\end{thebibliography}

%\newpage


%\clearpage
%\newpage
%\begin{figure}
%\includegraphics[width=15.0cm]{ky_combo.ps}
%\caption{}
%\label{figure:ky_combo}
%\end{figure}






\end{document}
